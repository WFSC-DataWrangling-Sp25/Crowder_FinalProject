% Options for packages loaded elsewhere
\PassOptionsToPackage{unicode}{hyperref}
\PassOptionsToPackage{hyphens}{url}
%
\documentclass[
]{article}
\usepackage{amsmath,amssymb}
\usepackage{iftex}
\ifPDFTeX
  \usepackage[T1]{fontenc}
  \usepackage[utf8]{inputenc}
  \usepackage{textcomp} % provide euro and other symbols
\else % if luatex or xetex
  \usepackage{unicode-math} % this also loads fontspec
  \defaultfontfeatures{Scale=MatchLowercase}
  \defaultfontfeatures[\rmfamily]{Ligatures=TeX,Scale=1}
\fi
\usepackage{lmodern}
\ifPDFTeX\else
  % xetex/luatex font selection
\fi
% Use upquote if available, for straight quotes in verbatim environments
\IfFileExists{upquote.sty}{\usepackage{upquote}}{}
\IfFileExists{microtype.sty}{% use microtype if available
  \usepackage[]{microtype}
  \UseMicrotypeSet[protrusion]{basicmath} % disable protrusion for tt fonts
}{}
\makeatletter
\@ifundefined{KOMAClassName}{% if non-KOMA class
  \IfFileExists{parskip.sty}{%
    \usepackage{parskip}
  }{% else
    \setlength{\parindent}{0pt}
    \setlength{\parskip}{6pt plus 2pt minus 1pt}}
}{% if KOMA class
  \KOMAoptions{parskip=half}}
\makeatother
\usepackage{xcolor}
\usepackage[margin=1in]{geometry}
\usepackage{color}
\usepackage{fancyvrb}
\newcommand{\VerbBar}{|}
\newcommand{\VERB}{\Verb[commandchars=\\\{\}]}
\DefineVerbatimEnvironment{Highlighting}{Verbatim}{commandchars=\\\{\}}
% Add ',fontsize=\small' for more characters per line
\usepackage{framed}
\definecolor{shadecolor}{RGB}{248,248,248}
\newenvironment{Shaded}{\begin{snugshade}}{\end{snugshade}}
\newcommand{\AlertTok}[1]{\textcolor[rgb]{0.94,0.16,0.16}{#1}}
\newcommand{\AnnotationTok}[1]{\textcolor[rgb]{0.56,0.35,0.01}{\textbf{\textit{#1}}}}
\newcommand{\AttributeTok}[1]{\textcolor[rgb]{0.13,0.29,0.53}{#1}}
\newcommand{\BaseNTok}[1]{\textcolor[rgb]{0.00,0.00,0.81}{#1}}
\newcommand{\BuiltInTok}[1]{#1}
\newcommand{\CharTok}[1]{\textcolor[rgb]{0.31,0.60,0.02}{#1}}
\newcommand{\CommentTok}[1]{\textcolor[rgb]{0.56,0.35,0.01}{\textit{#1}}}
\newcommand{\CommentVarTok}[1]{\textcolor[rgb]{0.56,0.35,0.01}{\textbf{\textit{#1}}}}
\newcommand{\ConstantTok}[1]{\textcolor[rgb]{0.56,0.35,0.01}{#1}}
\newcommand{\ControlFlowTok}[1]{\textcolor[rgb]{0.13,0.29,0.53}{\textbf{#1}}}
\newcommand{\DataTypeTok}[1]{\textcolor[rgb]{0.13,0.29,0.53}{#1}}
\newcommand{\DecValTok}[1]{\textcolor[rgb]{0.00,0.00,0.81}{#1}}
\newcommand{\DocumentationTok}[1]{\textcolor[rgb]{0.56,0.35,0.01}{\textbf{\textit{#1}}}}
\newcommand{\ErrorTok}[1]{\textcolor[rgb]{0.64,0.00,0.00}{\textbf{#1}}}
\newcommand{\ExtensionTok}[1]{#1}
\newcommand{\FloatTok}[1]{\textcolor[rgb]{0.00,0.00,0.81}{#1}}
\newcommand{\FunctionTok}[1]{\textcolor[rgb]{0.13,0.29,0.53}{\textbf{#1}}}
\newcommand{\ImportTok}[1]{#1}
\newcommand{\InformationTok}[1]{\textcolor[rgb]{0.56,0.35,0.01}{\textbf{\textit{#1}}}}
\newcommand{\KeywordTok}[1]{\textcolor[rgb]{0.13,0.29,0.53}{\textbf{#1}}}
\newcommand{\NormalTok}[1]{#1}
\newcommand{\OperatorTok}[1]{\textcolor[rgb]{0.81,0.36,0.00}{\textbf{#1}}}
\newcommand{\OtherTok}[1]{\textcolor[rgb]{0.56,0.35,0.01}{#1}}
\newcommand{\PreprocessorTok}[1]{\textcolor[rgb]{0.56,0.35,0.01}{\textit{#1}}}
\newcommand{\RegionMarkerTok}[1]{#1}
\newcommand{\SpecialCharTok}[1]{\textcolor[rgb]{0.81,0.36,0.00}{\textbf{#1}}}
\newcommand{\SpecialStringTok}[1]{\textcolor[rgb]{0.31,0.60,0.02}{#1}}
\newcommand{\StringTok}[1]{\textcolor[rgb]{0.31,0.60,0.02}{#1}}
\newcommand{\VariableTok}[1]{\textcolor[rgb]{0.00,0.00,0.00}{#1}}
\newcommand{\VerbatimStringTok}[1]{\textcolor[rgb]{0.31,0.60,0.02}{#1}}
\newcommand{\WarningTok}[1]{\textcolor[rgb]{0.56,0.35,0.01}{\textbf{\textit{#1}}}}
\usepackage{graphicx}
\makeatletter
\def\maxwidth{\ifdim\Gin@nat@width>\linewidth\linewidth\else\Gin@nat@width\fi}
\def\maxheight{\ifdim\Gin@nat@height>\textheight\textheight\else\Gin@nat@height\fi}
\makeatother
% Scale images if necessary, so that they will not overflow the page
% margins by default, and it is still possible to overwrite the defaults
% using explicit options in \includegraphics[width, height, ...]{}
\setkeys{Gin}{width=\maxwidth,height=\maxheight,keepaspectratio}
% Set default figure placement to htbp
\makeatletter
\def\fps@figure{htbp}
\makeatother
\setlength{\emergencystretch}{3em} % prevent overfull lines
\providecommand{\tightlist}{%
  \setlength{\itemsep}{0pt}\setlength{\parskip}{0pt}}
\setcounter{secnumdepth}{-\maxdimen} % remove section numbering
\ifLuaTeX
  \usepackage{selnolig}  % disable illegal ligatures
\fi
\usepackage{bookmark}
\IfFileExists{xurl.sty}{\usepackage{xurl}}{} % add URL line breaks if available
\urlstyle{same}
\hypersetup{
  pdftitle={Final Project Proposal},
  pdfauthor={Ellen Bledsoe},
  hidelinks,
  pdfcreator={LaTeX via pandoc}}

\title{Final Project Proposal}
\author{Ellen Bledsoe}
\date{2025-04-04}

\begin{document}
\maketitle

\section{Final Project Proposal}\label{final-project-proposal}

Answering the questions in this document are worth 20 points for
Question 5 in Week 9's Assignment.

\subsection{Where to Find Data}\label{where-to-find-data}

\textbf{If you have your own data from a research project, use it!}

If you don't have your own data, here are some suggestions for where to
find data:

\begin{enumerate}
\def\labelenumi{\arabic{enumi}.}
\tightlist
\item
  From your lab group
\end{enumerate}

If you are working in a lab group, you can almost certainly find data to
work with from your lab! Ask your PI or other lab members if you can use
their data for this project. It is often helpful to have data that
relates somewhat to a project or topic you are familiar with.

\begin{enumerate}
\def\labelenumi{\arabic{enumi}.}
\setcounter{enumi}{1}
\tightlist
\item
  From the ``Data is Plural'' Archive
\end{enumerate}

\href{https://www.data-is-plural.com/}{Data is Plural} is a weekly
newsletter about complied datasets that people have found. They have an
archive of all of the datasets that have been included in the
newsletters, which is nearly 2000.

Just because the datasets have been compiled does not mean that they are
clean! I've used a few of them in class as demonstrations.

\begin{enumerate}
\def\labelenumi{\arabic{enumi}.}
\setcounter{enumi}{2}
\tightlist
\item
  From an agency, NGO, or non-profit
\end{enumerate}

If you have contacts in a local, state, or federal agency or non-profit,
they likely have some languishing data floating around that you can
clean up.

\begin{enumerate}
\def\labelenumi{\arabic{enumi}.}
\setcounter{enumi}{3}
\tightlist
\item
  A data repository
\end{enumerate}

There are many data repositories where you can find data. I would
recommend sticking to some of the other options above to keep your
search a bit more constrained, but feel free to ultimately use any
dataset you find that you would like to use.

\begin{enumerate}
\def\labelenumi{\arabic{enumi}.}
\setcounter{enumi}{4}
\tightlist
\item
  From me!
\end{enumerate}

If you aren't having any success finding a dataset, let me know. I have
a few I can share with you, if needed.

\subsection{Project Guidelines}\label{project-guidelines}

As a reminder, the general guidelines for the final project that the
project will need to include core concepts covered in a certain number
of weeks (6 for WFSC 496B, 10 for WFSC 596B), bringing together many
aspects of what we will have learned through the course of the semester.

The final project for this course is worth 300 points (WFSC 496B) or 500
points (WFSC 596B).

\subsection{Questions}\label{questions}

The following answers are not binding! You can change your final project
as much as you'd like from what you propose here.

Answering these questions is meant to get you thinking about what data
you would like to use and what tasks you might perform with the data.

Note: Your data set does not need to be ecological in nature if you have
another interest. It should also not be a dataset that we have used in
class (see me if you have questions).

\begin{enumerate}
\def\labelenumi{\arabic{enumi}.}
\item
  What data are you planning to use for the final project? Describe the
  data: its contents, its format(s), one or more dataframes, etc. If you
  have a link to the dataset, please include it. If you have a file that
  you can upload, please add it to the assignment submission.
\item
  Does this data need to be cleaned in any way to make it tidy? If so,
  what needs to occur?
\item
  Speaking generally, what tasks will you perform with this dataset? It
  might include tasks that we have not yet covered how to accomplish in
  class, but come up with a general plan that you would like to follow.
\item
  One of the requirements of the final project is to make some type of
  plot using \texttt{ggplot2} with the data. What do you plan to plot
  from your data?
\end{enumerate}

\emph{Answer}:

\begin{Shaded}
\begin{Highlighting}[]
\CommentTok{\#1}
\CommentTok{\#the data that I am planning on using is a text file called e249\_Plant aboveground biomass data. I downloaded it off of EDI portal. It is from research focusing on General climate models (GCMs). The Biodiversity and Climate (BAC) experiment is focusing on the "direct and interactive effects" of plant species numbers, temperature, community composition, precipitation, and plant species numbers on "11 productivity, C and N dynamics, stability, and plant, microbe, and insect species abundances in CDR grassland ecosystems". I can only assume "CDR" means "Cedar Creek Long Term Ecological Research (CDR LTER)" because that\textquotesingle{}s all I was able to find on it. This was created by David Tilman.}
\CommentTok{\#2}
\CommentTok{\#it for sure needs to be cleaned. It has messy dates and column names, the information in the rows is inconsistent, the entire date column is incorrectly formatted, it\textquotesingle{}s extrememly long, data is described in the cells as mischellaneous or unsorted, etc.. It\textquotesingle{}s not great. }
\CommentTok{\#3}
\CommentTok{\#I\textquotesingle{}ll have to work on finding what is useful data and what is basically NA data. I\textquotesingle{}ll rename the columns and sort through the date columns. I\textquotesingle{}d also have to figure out how to arrange it so that it\textquotesingle{}s neat and create new columns to house the extra information that was left out or jammed into other cells.}
\CommentTok{\#4}
\CommentTok{\# I am currently planning to make a bar plot or line plot of the heat treatment and mass by year. With time, I could potentially do it by species but I\textquotesingle{}m not sure how much time that would take. }
\end{Highlighting}
\end{Shaded}

\subsection{Submitting this File}\label{submitting-this-file}

\begin{enumerate}
\def\labelenumi{\arabic{enumi}.}
\tightlist
\item
  Create a new repository in the class GitHub organization for your
  final project. Use the naming convention ``LastName\_FinalProject''
\item
  Connect the repository to an RStudio project
\item
  Create a series of appropriate sub-directories in your RStudio
  Project, including a \texttt{documents} folder.
\item
  Move this file to the documents folder.
\item
  Stage, commit, and push this document (and sub-directory) to GitHub
  with a descriptive commit message.
\end{enumerate}

\end{document}
